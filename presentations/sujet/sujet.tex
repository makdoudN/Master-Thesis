\section{Objectifs et Défis technologiques}
\begin{frame}{Objectifs du stage}
\textbf{Sujet du stage}

Contrôler un agent dans un environnement simulé à partir de la vision de celui-ci.
\end{frame}

\begin{frame}{Contraintes sur le contrôle}

Contrôler un agent dans un environnement simulé à partir de la vision de celui-ci.

\begin{alertblock}{Contrainte sur notre contrôle}
    \begin{itemize}
        \item Etre pertinant pour toutes entrées (images, graphes, ...) .
        \item Etre approprié pour tous buts.
        \item Gérer des environnements complexes et inconnues.
    \end{itemize}
\end{alertblock}

\end{frame}

\begin{frame}{Technologies utilisées et contributions}
    \begin{alertblock}{Contraintes sur notre contrôle}
    \begin{itemize}
        \item Etre pertinant pour toutes entrées (images, graphes, ...) .
        \item Etre approprié pour tous buts.
        \item Gérer des environnements complexes et inconnues.
    \end{itemize}
\end{alertblock}

\begin{exampleblock}{Technologiee utilisées pour répondre à ce défi}
    \begin{itemize}
        \item \textbf{L'apprentissage profond} pour la résilence fâce aux entrées.
        \item \textbf{L'apprentissage par renforcement} pour s'adapter aux buts.
        \item \textbf{Module de curiosité} pour explorer.
    \end{itemize}
\end{exampleblock}
\end{frame}

\begin{frame}{La contribution de ce stage}
\begin{block}{Conclusion sur notre contribution}
    Proposition d'un contrôle  basé sur de l'apprentissage par renforcement profond utilisant un module de curiosité.
\end{block}


\end{frame}
