% -- simple test

\newglossaryentry{A3C}
{
    name=A3C,
    description={Acronyme pour  Asynchronous Actor-Critic Agents qui est l'algorithme qui servira de base pour le contrôle de notre agent.}
}


\newglossaryentry{IA}
{
    name=IA,
    description={Acronyme d'intelligence artificielle, cela désigne un ensemble de technologie visant à un apprentissage automatique. En particuliers, pendant ce stage, nous explorerons des sous-parties de l'intelligence artificielle qui sont l'apprentissage profond et l'apprentisage par renforcement.}
}


\newglossaryentry{API}
{
    name=API,
    description={En anglais: \emph{Application program interface}, c'est un ensemble de protocoles, ou de contrats spécifiant le fonctionnant d'un programme. Dans le cas de ce rapport, cela correspond à un contrat entre l'application client et serveur (SE-STAR)}
}

\newglossaryentry{RL}
{
    name=RL,
    description={RL est l'acronyme de reinforcement learning qui en francais se traduit par apprentissage par renforcement.}
}


\newglossaryentry{GPU}
{
    name=GPU,
    description={En anglais: \emph{Graphics Processing Unit}, est un composant normalement utilisé dans la gestion des graphismes qui permet d'effectuer des calcules hautement parallèlisable de façon extrêmement rapide}
}

\newglossaryentry{PAAC}
{
    name=PAAC,
    description={Référence à une méthode en apprentissage par renforcement nommée \emph{Efficient Parallel Methods for Deep Reinforcement Learning }\cite{2017arXiv170504862C}. Elle se repose sur la même stratégie que l'A3C en utilisant une architecture synchrone permettant l'utilisation du GPU} 
}

\newglossaryentry{Wine}
{
    name=Wine,
    description={En anglais: \emph{Wine Is Not an Emulator}, est une application permettant l'utilisation d'application Windows sous Linux. Son nom vient du faite que de nombreuses personnes pensent a tort que Wine est un 'emulateur de Windows' or Wine a réimplémenté l'ensemble des appels systèmes dans le user space linux permettant à l'utilisateur de faire fonctionner une application Windows sous une distriubtion Linux}
}


\newglossaryentry{framework}
{
    name=framework,
    description={Un framework est un logiciel proposant une architecture unifiée répondant à une problématique. Dans ce rapport, on pourra parler de framework de deep learning qui sont des outils proposant une API dans la réalisation de nos algorithmes.}
}

\newacronym{gcd}{GCD}{Greatest Common Divisor}


